\section{Функциональное тестирование}

Функциональные тесты основываются на функциях, выполняемых системой. Как правило, эти функции описываются в требованиях, функциональных спецификациях или в виде случаев использования системы (use cases).

\subsection{Use Cases}

\textbf{Use Case} (вариант использования, ВИ, Прецедент, юскейс) — это сценарная техника описания взаимодействия. С помощью Use Case может быть описано и пользовательское требование, и требование к взаимодействию систем, и описание взаимодействия людей и компаний в реальной жизни. В общем случае, с помощью Use Case может описываться взаимодействие двух или большего количества участников, имеющее конкретную цель. В разработке ПО эту технику часто применяют для проектирования и описания взаимодействия пользователя и системы, поэтому название Use Case часто воспринимает как синоним требования человека-пользователя к решению определенной задачи в системе.


Примеры Use Case для тестируемого приложения:
\begin{enumerate}
	\item создание пользователем команды
	\item создание пользователем федерации
	\item создание пользователем турнира по футболу
	\item создание пользователем матчей в турнире его федерации
	\item подача пользователем заявки на участе его команды в турнире 
\end{enumerate}

\subsection{Use Case создание пользователем команды}

Рассмотрим подробнее use case создание пользователем футбольной команды.

\textbf{Система} - сайт web-приложения

\textbf{Основное действующее лицо} - пользователь-капитан команды

\textbf{Цель} - создать команду

\textbf{Триггер} - пользователь решает создать команду и заходит на главную страницу сайта

\textbf{Результат} - информация о команде сохранена в базе данных, пользователь получил ответ от сервера.

\subsubsection{Основной поток событий}
\begin{enumerate}
	\item пользователь заходит на главную страницу сайта
	\item пользователь авторизуется на сайте через сайт-вконтакте
	\item пользователь нажимает на кнопку <<создать команду>>
	\item пользователь заполняет поля <<название команды>> и <<город>>
	\item пользователь нажимает отправить
	\item система получает запрос
	\item система проверяет запрос
	\item при верном запросе система сохраняет данные в базе
	\item система посылает ответ пользователю
\end{enumerate}

\subsection{Автоматизация функционального тестирования}

В связи с большим колличеством usecase и большим колличеством монотонных действий в каждом usecase достаточно распространенной является автоматизация функционального тестирования.

Автоматизация функционального тестирования выполнена с помощью Selenium WebDriver.  

\textbf{Selenium WebDriver} - автоматизирует действия с браузером. Это именно то, что необходимо для тестирования сайта web-приложения. Частым является использование паттерна Page Object при написании тестов с использованием Selenium.

\subsubsection{Тестирование создания команды}

Код теста написан с учетом рассмотренного потока событий данного usecase в листинге \ref{lst:test-create-team}

\begin{lstlisting}[caption={Тестирование use case создание команды}, label={lst:test-create-team}]
it('create team', function(done) {
        client.manage().timeouts().implicitlyWait(5000);

        let teamName = 'NAME TEAM1';
        let teamCity = 'CITY TEAM1';

        vkPage.open()
            .then(() => {
                vkPage.auth(secret.emailVk, secret.passVk);
                mainPage.open(URL);
                mainPage.auth();
                mainPage.createTeam(teamName, teamCity);
                client.quit();
            })
            .then(() => {
                return Team.find({name: teamName}, function (err, team) {
                    assert.isNull(err);
                    assert.isNotNull(team);
                    done();
                })
            })
\end{lstlisting} 

\begin{lstlisting}[caption={Объект mainPage}]
function mainPage(driver) {
    let self = this;

    self.driver = driver;

    self.open = function(url) {
        return self.driver.get(url);
    };

    self.auth = function() {
        self.driver.findElement(wd.By.id('vk-auth-btn')).click();

        self.driver.findElements(wd.By.className("button_indent"))
            .then(arr => {
                if (arr.length > 0) {
                    arr[0].click();
                }
            });
    };

    self.createTeam = function(teamName, teamCity) {
        // var createTeamBtn = self.driver.wait(wd.until.elementLocated(wd.By.id('create-team-btn')), 10000);
        // createTeamBtn.click();
        self.driver.findElement(wd.By.id('create-team-btn')).click();

        self.driver.findElement(wd.By.id('team-name')).sendKeys(teamName);
        self.driver.findElement(wd.By.id('team-city')).sendKeys(teamCity);

        self.driver.findElement(wd.By.id('send-team-btn')).click();
    }

}
\end{lstlisting} 

\begin{lstlisting}[caption={Результаты тестирования}]
ilyaps@debian-ilyaps:~/projects/MyFootballFederation$ npm test

> mff@1.0.0 test /home/ilyaps/projects/MyFootballFederation
> mocha test/gui/ --no-timeouts


запущенная база: mongodb://localhost:27017/football_test


Express запущен на http://localhost:8080; нажмите Ctrl+C для завершения.
  GUI Create team
    ОК normal create team (21944ms)


  1 passing (22s)
\end{lstlisting} 


